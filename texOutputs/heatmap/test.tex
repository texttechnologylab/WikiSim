\RequirePackage{luatex85}
\documentclass{article}
\usepackage{tikz}
\usepackage{collcell}
\usepackage[no-math]{fontspec}
\usepackage[lm]{sfmath}
 \usetikzlibrary{positioning}
\usepackage[active,tightpage]{preview}
\PreviewEnvironment{tikzpicture}
\begin{document}%\footnotesize
 
 \definecolor{SeminarRot}{rgb}{0.75,0,0}
 \definecolor{SeminarGruen}{RGB}{0,128,0}
 \definecolor{SeminarOrange}{RGB}{237,167,45} % rgb: {0.93,0.65,0.18}
  \definecolor{SeminarBlau}{RGB}{0,154,224} % rgb: {0.93,0.65,0.18} 
 \definecolor{Braun}{RGB}{140,86,75} % rgb: {0.93,0.65,0.18}   
 %%% from http://texblog.org/2013/06/13/latex-heatmap-using-tabular/
 % provide min and max numbers for coloring mixing
 \newcommand*{\MinNumber}{1.0}
 \newcommand*{\MaxNumber}{0}
 % third value comes from table entry
 
 % color mixing macro:
 \makeatletter
 \newcommand{\ApplyGradient}[1]{%
  \ifdim\dimexpr#1pt=\@ne pt%
  \pgfmathsetmacro{\PercentColor}{100.0*(#1-\MinNumber)/(\MaxNumber-\MinNumber)}
  \hspace{-0.33em}\colorbox{SeminarGruen}{}
  \else
  \pgfmathsetmacro{\PercentColor}{100.0*(#1-\MinNumber)/(\MaxNumber-\MinNumber)}
  \hspace{-0.33em}\colorbox{white!\PercentColor!SeminarGruen}{}
  \fi
 }
 \makeatother
 
 % two color mixing macro, distinguishing positive from negative number:
 % positive: green
 % negative: red
 \makeatletter
 \newcommand{\ApplyTwoGradients}[1]{%
   \ifdim\dimexpr#1pt=\thr@@ pt
   \pgfmathsetmacro{\PercentColor}{100.0*(#1-\MinNumber)/(\MaxNumber-\MinNumber)}
   \hspace{-0.33em}\colorbox{Braun}{}
   \else
	\ifdim\dimexpr#1pt=\tw@ pt
   \pgfmathsetmacro{\PercentColor}{100.0*(#1-\MinNumber)/(\MaxNumber-\MinNumber)}
   \hspace{-0.33em}\colorbox{SeminarBlau}{}
   \else
   \ifdim\dimexpr#1pt=\z@%
   \pgfmathsetmacro{\PercentColor}{100.0*(#1-\MinNumber)/(\MaxNumber-\MinNumber)}
   \hspace{-0.33em}\colorbox{white}{}
   \else
   \ifdim\dimexpr#1pt>\z@
   \pgfmathsetmacro{\PercentColor}{100.0*(#1-\MinNumber)/(\MaxNumber-\MinNumber)}
   \hspace{-0.33em}\colorbox{white!\PercentColor!SeminarGruen}{}
   \else
   \pgfmathsetmacro{\PercentColor}{100.0*(-#1-\MinNumber)/(\MaxNumber-\MinNumber)}
   \hspace{-0.33em}\colorbox{white!\PercentColor!SeminarRot}{}
   \fi\fi\fi\fi
 } \makeatother 
 % attach macro to column type:
 \newcolumntype{R}{>{\collectcell\ApplyGradient}c<{\endcollectcell}}
 \newcolumntype{T}{>{\collectcell\ApplyTwoGradients}c<{\endcollectcell}}
 
 % remove spacing in tabular:
 \renewcommand{\arraystretch}{0}
 \setlength{\tabcolsep}{0pt}
 
 % set size of color boxes
 \setlength{\fboxsep}{2.5mm}
 
 \newcommand{\mcvert}[1]{\multicolumn{1}{c}{\rotatebox{90}{\space #1}}}
 \newcommand{\mchori}[1]{\multicolumn{1}{l}{\raisebox{-4pt}{#1}\rule{1em}{0pt}}}
\newcounter{colno}
\setcounter{colno}{100}
\stepcounter{colno}

\colorlet{colorone}{SeminarGruen}
\colorlet{colortwo}{SeminarRot}

\begin{tikzpicture}
\node [inner sep=0pt] (tab) {
\begin{tabular}{rRRRRRRRRRRRR}
 &  \mcvert{ceb} & \mcvert{de} & \mcvert{es} & \mcvert{fr} & \mcvert{it} & \mcvert{ja} & \mcvert{nl} & \mcvert{pl} & \mcvert{ru} & \mcvert{sv} & \mcvert{vi} & \mcvert{war} \\
\mchori{ceb}  &  0.0  &  0.05  &  0.07  &  0.55  &  0.46  &  0.06  &  0.18  &  0.34  &  0.34  &  0.04  &  0.0  &  0.0 \\
\mchori{de}  &  0.05  &  0.0  &  0.12  &  0.6  &  0.51  &  0.11  &  0.23  &  0.39  &  0.4  &  0.09  &  0.06  &  0.05 \\
\mchori{es}  &  0.07  &  0.12  &  0.0  &  0.62  &  0.52  &  0.12  &  0.25  &  0.4  &  0.41  &  0.11  &  0.07  &  0.07 \\
\mchori{fr}  &  0.55  &  0.6  &  0.62  &  0.0  &  1.0  &  0.61  &  0.73  &  0.89  &  0.89  &  0.59  &  0.56  &  0.55 \\
\mchori{it}  &  0.46  &  0.51  &  0.52  &  1.0  &  0.0  &  0.51  &  0.64  &  0.79  &  0.8  &  0.5  &  0.46  &  0.46 \\
\mchori{ja}  &  0.06  &  0.11  &  0.12  &  0.61  &  0.51  &  0.0  &  0.24  &  0.39  &  0.4  &  0.1  &  0.06  &  0.06 \\
\mchori{nl}  &  0.18  &  0.23  &  0.25  &  0.73  &  0.64  &  0.24  &  0.0  &  0.52  &  0.53  &  0.22  &  0.19  &  0.18 \\
\mchori{pl}  &  0.34  &  0.39  &  0.4  &  0.89  &  0.79  &  0.39  &  0.52  &  0.0  &  0.68  &  0.38  &  0.34  &  0.34 \\
\mchori{ru}  &  0.34  &  0.4  &  0.41  &  0.89  &  0.8  &  0.4  &  0.53  &  0.68  &  0.0  &  0.39  &  0.35  &  0.34 \\
\mchori{sv}  &  0.04  &  0.09  &  0.11  &  0.59  &  0.5  &  0.1  &  0.22  &  0.38  &  0.39  &  0.0  &  0.04  &  0.04 \\
\mchori{vi}  &  0.0  &  0.06  &  0.07  &  0.56  &  0.46  &  0.06  &  0.19  &  0.34  &  0.35  &  0.04  &  0.0  &  0.0 \\
\mchori{war}  &  0.0  &  0.05  &  0.07  &  0.55  &  0.46  &  0.06  &  0.18  &  0.34  &  0.34  &  0.04  &  0.0  &  0.0 \\
\end{tabular}
};
\end{tikzpicture}

\end{document}
